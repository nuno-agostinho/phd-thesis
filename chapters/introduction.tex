% !TEX root = ../PhD Thesis.tex
\chapter{Introduction}
\pagenumbering{arabic}

\section{On the origin of life}

This is the story of my PhD, a personal journey like no other I have faced before. To follow my story, we have to rewind back to years ago. Billions and billions of years ago. Once upon a time there was a violent, harsh and unwelcoming planet among countless others. Earth was lifeless. But as millions of years went by, it started being home to a complex recipe whose special sauce is still being studied to this day: the primordial soup. These were the perfect conditions for a young, 500-million-year-old planet to brew life.

And what is life? Although this question is not easy to answer, living organisms as we know them are complex, carbon-based systems composed of nucleic acids, proteins, carbohydrates and lipids. Together with some smaller molecules, these molecules are known as biomolecules and are crucial for the survival of living organisms.

Amongst those biomolecules, two are particularly relevant to my story: proteins and nucleic acids. Proteins have many important functions in an organism, including catalysing chemical reactions (enzymes), signalling cellular processes (hormones) and playing a role in the immune system (antigens). Regarding nucleic acids, deoxyribonucleic acid (DNA) stores the genetic data, the blueprint required to generate the majority of the vital molecules in the cell, including ribonucleic acid (RNA) molecules for protein synthesis and regulation.
% messenger RNAs (mRNA) are coding RNAs whose sequence is used as a template to create new proteins, transfer RNAs (tRNA) carry amino acids to generate proteins and ribosomal RNAs (rRNA) are the major components of the ribosome, the protein synthesis machinery. Other non-coding RNAs (long non-coding RNAs, small interfering RNAs, micro RNAs, etc.) have further regulatory roles in the cell.

% Miller-Urey experiment
% But how did all this complexity emerged from the primordial soup? The primitive Earth's  atmosphere was rich in inorganic compounds (the exact ones are still up to debate). Together with 

One possibility for the origin of life is based on the idea of %that ribonucleic acid (RNA) was the molecule that made (modern\footnote{Other transitory living organisms that precede RNA-based living organisms are also hypothesised.}) life possible:
the RNA world, an hypothesis in which self-replicating RNA evolutionarily predates DNA and proteins \cite{gilbert:1986td}. After all, RNA molecules are able to store genetic information like DNA and some can even catalyse life-critical chemical reactions like enzymes, making RNA a prime candidate for life to take its first steps \cite{gilbert:1986td}. Later on, these specific functions may have been overtaken by enzymes, proteins that were more effective as reaction catalysers, and DNA, a more stable and less error-prone nucleic acid to store genetic information \cite{gilbert:1986td}.

% Re-creating the RNA world, 1995 review, https://www.cell.com/action/showPdf?pii=S0960-9822%2895%2900205-3

% Progenote: the last common ancestor of modern life

% Gene expression

\section{On the origin of species}

Throughout millions of years, evolution continued.

\section{On nucleic acids and protein synthesis}

It was in the year of 1869 that Friedrich Miescher isolated a mysterious, protein-like substance that he named \emph{nuclein}, found in the cell nucleus of diverse vertebrates. Miescher's work led him to believe that an increase in nuclein could be associated with the first stages of cell division in proliferating tissues \cite{dahm:2005wx}. Nuclein was renamed nucleic acid in 1889 \cite{dahm:2005wx}.

% 1885-1901, isolated the 5 nucleobases that constitute DNA + RNA

% 1902–1909: Archibald Garrod proposes that genetic defects result in the loss of enzymes and hereditary metabolic diseases.

Contrary to the consensus in the first decades of the 20th century, Boveri and Sutton theorised that the chromosomes -- not proteins as previously thought -- carried genetic information \cite{sutton:1902tx,dahm:2005wx}. According to Sutton:

\begin{displayquote}[\cite{sutton:1902tx}]
the association of paternal and maternal chromosomes in pairs and their subsequent separation during the reducing division (...) may constitute the physical basis of the Mendelian law of heredity.
\end{displayquote}

The Boveri-Sutton chromosome theory of genetic inheritance followed Gregor Mendel's controversial \cite{} work in 1865 \cite{sutton:1902tx} and was later supported by fruit fly experiments from an initially skeptic \cite{} Thomas Morgan \cite{morgan:1915tw}. In 1915, Morgan and colleagues published a textbook with their findings describing genetic dominance, sex inheritance and chromosomal crossover. One chapter was provokingly titled \emph{The Chromosomes as Bearers of Hereditary Material} \cite{morgan:1915tw}.

% 1928: Frederick Griffith postulates that a “transforming principle” permits properties from one type of bacteria (heat-inactivated virulent Streptococcus pneumoniae) to be transferred to another (live nonvirulent Streptococcus pneumoniae).

% Around 1930s, Phoebus Levene identified the four DNA nucleotides (adenine, cytosine, guanine, and thymine), as well as the sugar-phosphate backbone \cite{levene:1929ug}. % confirm if in this paper

% In 1933, Jean Brachet found evidence of \emph{thymus nucleic acid} (nowadays known as DNA) in the cell nucleus and of \emph{yeast nucleic acid} (RNA) in the cytoplasm\footnote{At the time, \emph{thymus nucleic acid} was thought to be a nucleic acid from animals (specially found in the thymus, hence its name) and \emph{yeast nucleic acid} from plants.}.

Although the word \emph{gene} was popularly used since being coined by Johannsen in 1909 to abstractly refer to Mendelian factors of inheritance (i.e. the units of heredity) \cite{}, Demerec tried to define its concept in his 1933 publication, \emph{What is a Gene?}: % Demerec also depicts a tentative structure of DNA

\begin{displayquote}[\cite{}]
(...) [A gene] is a minute organic particle, capable of reproduction, located in a chromosome and responsible for the transmission of a hereditary characteristic.
\end{displayquote}

Later in 1941, George Beadle and Edward Tatum hypothesised that each gene is responsible for producing a specific enzyme and demonstrated that radiation-induced mutations could alter the resulting enzyme.

% 1949: Colette and Roger Vendrely and André Boivin discover that the nuclei of germ cells contain half the amount of DNA that is found in somatic cells. This parallels the reduction in the number of chromosomes during gametogenesis and provides further evidence for the fact that DNA is the genetic material.

% 1952: Alfred Hershey and Martha Chase use viruses (bacteriophage T2) to confirm DNA as the genetic material by demonstrating that during infection viral DNA enters the bacteria while the viral proteins do not and that this DNA can be found in progeny virus particles.

% In 1953, the work by Watson, Crick, Rosalind Franklin, Maurice Wilkins on DNA's double helix structure
% Molecular Structure of Nucleic Acids, Watson + Crick 1953, http://dosequis.colorado.edu/Courses/MethodsLogic/papers/WatsonCrick1953.pdf

In 1955, George Palade described the ribosome as "a small particulate component of the cytoplasm" that associates with RNA in the endoplasmic reticulum membrane to perform protein synthesis \cite{palade:1955tf,jacob:1961uh}. The associated RNA was divided in two: ribosomal RNA (rRNA) that composed the ribosome itself and \emph{soluble RNA} -- transfer RNA (tRNA) --, found to carry the amino acids for protein synthesis \cite{hoagland:1958vm,jacob:1961uh}.

In 1956, the DNA polymerase is discovered, an enzyme that replicates DNA. 

In 1957, Francis Crick proposes that the genetic information flows from DNA to protein via RNA: the \emph{central dogma of molecular biology}.
% There was once a time when scientists did not all agree on the idea that nucleic acids played a role in protein synthesis.
Crick also proposed in 1958 that triplets (\emph{codons}) of the four nucleotides found in nucleic acids were necessary to produce each of the 20 universally-found types of amino acids that compose a protein \cite{crick:1958ws,crick:1961ui} and that the amino acids would be responsible for the protein's three-dimensional structure -- and consequently, its functionality \cite{crick:1958ws}.

In 1960, DNA-dependent RNA polymerase, an enzyme that synthesises RNA from DNA and common to all living organisms, was independently described.

François Jacob and Jacques Monod speculated in 1961 that ribosomal protein synthesis required an intermediate molecule with the template message to convert from DNA to protein and that would act as the \emph{messenger} \cite{jacob:1961uh,brenner:1961ve}. Unlike many of their contemporaries, they dismissed rRNA (and tRNA) molecules as the template for protein synthesis, given that they did not reflect the base composition of DNA, among other properties \cite{jacob:1961uh}. There were some published experiments on unstable RNA molecules with distinct properties from rRNA and tRNA, which Jacob and Monod proposed as relevant to their hypothesis and categorised them as messenger RNA (mRNA) \cite{jacob:1961uh,brenner:1961ve}.

% General Nature of the Genetic Code for Proteins \cite{crick:1961ui}

% Multiple ribosomes were later found to bind to a single RNA molecule (polysomes), allowing for parallelised protein synthesis \cite{warner:1963uj}.

% 1951-1965: Different kinds of tRNA were identified in the cell, each associated with a single specific amino acid. tRNAs have sequences complementary to mRNA codons and they are required for the next correct amino acid during protein synthesis.

% 1961–1966: Genetic code cracked by Robert W. Holley, Har Gobind Khorana, Heinrich Matthaei, Marshall W. Nirenberg, and colleagues: https://pubmed.ncbi.nlm.nih.gov/5322508/

1971: mRNA has a poly-A tail.

1966-75: RNA is processed by adding a polyA-tail and a 5' cap.

% 1968: The origin of the genetic code, Crick, https://doi.org/10.1016/0022-2836(68)90392-6

% Most RNAs are non-coding: ~97% in eukaryotes?

\section{On alternative splicing}

First reported in mammalian cells infected with a human adenovirus 2 \cite{berget:1977wp,chow:1977wn} and later observed in endogenous mammalian and eukaryotic genes \cite{}, mRNA-DNA hybridisation experiments suggested that genes are composed by intervening non-coding sequences. During transcription of the precursor mRNA (pre-mRNA), the non-coding sequences (introns) are excised, in contrast with the expressed segments (exons), in a process called RNA splicing \cite{berget:1977wp,chow:1977wn,gilbert:1978wr}. In 1985, a RNA-protein complex composed by U1, U2, U4, U5 and U6 small nuclear ribonucleoproteins (snRNPs) was reported crucial for RNA splicing: the spliceosome \cite{grabowski:1985vm}.

The spliceosome catalyses the removal of introns from pre-mRNA in two transesterification steps: (1) the 5' end of the intron is cleaved and united to the conserved adenosine in the branch point sequence, forming an intermediary intron lariat, and then (2) the 3' end of the intron is cleaved, releasing the intron lariat, and the two flanking exons are ligated \cite{grabowski:1985vm,ruskin:1985vl,horowitz:1993wq}. The intron lariat is debranched (i.e. converted to a linear form) before its degradation \cite{ruskin:1985vl,arenas:1987vc}.

Introns are recognised by the spliceosome via the 5' and 3' splice sites (exon-intron junctions) and the branch point sequence and polypyrimidine tract (within the intronic region).

However, RNA splicing may excise different sequences depending on its regulation: alternative splicing.

\subsection{On sequencing}

% 1977: Frederick Sanger, Allan Maxam, and Walter Gilbert develop methods to sequence DNA.

% 1983: Kary Mullis invents PCR as a method for amplifying DNA in vitro.

From 1995 to 2000, the genomes of multiple organisms are published, including the bacterium \emph{H. influenzae}, the yeast \emph{S. cerevisiae}, the nematode \emph{C. elegans}, the fruit fly \emph{Drosophila}, and the plant Arabidopsis. In 2001, the human genome is finally published since the project started in 1990.

\section{On alternative splicing}

A gene is a segment of DNA that is transcribed to RNA and, in case of mRNA, may later be encoded as a protein. The whole process from gene to its product is known as gene expression and summarises multiple, complex steps that occur within the cell to maintain its well-being. Such processes include RNA synthesis (transcription), RNA splicing and protein synthesis (translation).

It was also Crick that suggested we would study evolution by comparing sequences across species.

% protein-coding genes vs. complexity

% Going back to today, RNAs are transcribed from DNA segments by the RNA polymerase enzyme in all living organisms. In eukaryotic cells, these RNAs are also processed via polyadenylation, 5' capping and splicing, during or after transcription.

Alternative splicing (AS) is a process where different RNA sequences can be produced from a single gene, promoting transcriptome diversity. The most extraordinary example reported is the Dscam gene in \emph{Drosophila melanogaster} (fruit fly) with more than 30 000 alternative transcripts reported to date. The multiple isoforms of this gene play a role the immune system of the fruit fly and may lead to more antigen diversity, thus increasing evolutionary flexibility.
% how many functional proteins? what more about this example? maybe useful to talk about other topic such as trans/cis-acting elements or spliceosome?

The splicing of those multiple isoforms is regulated via the interplay between RNA-binding proteins (RBPs) -- trans-acting regulators --, and the intronic or exonic regions of the transcript where they bind to -- cis-acting sequences. Different cis-acting sequences may act as either splicing enhancers or inhibitors. This differential regulation has been studied across cell types, development stages and tissues.

% On the origin of RNA splicing and introns, Phil Sharp 1985, https://doi.org/10.1016/0092-8674(85)90092-3

Multiple types of alternative splicing have been described, including skipped exons, mutually exclusive exons, alternative 5' and 3' splice sites and intron retention.

% When first discussing the RNA world in 1986, Gilbert proposed that RNA splicing may have been important to promote the evolutionary potential of RNAs \cite{gilbert:1986td}.

AS conservation across eukaryotes.

AS is deregulated in multiple disease contexts, including cancer and neurodegeneration. Multiple hallmarks of cancer are related with changes in splicing factors.

% The stress of exams on medical students causes AS changes in SMG1 that may have consequent effects on nonsense-mediated RNA decay and the p53 pathway (Kurokawa et al., 2010).

AS has been recently in the news because of being a therapeutic target. For instance, AS changes in gene X can be targeted. % Adrian Krainer

\section{Bioinformatics}

Following the sequencing of insulin by Sanger, the technique started being applied to study the amino acids of other proteins.

% A Protein Sequenator: https://febs.onlinelibrary.wiley.com/doi/full/10.1111/j.1432-1033.1967.tb00047.x

Dayhoff was one of the first scientists to compile known protein sequences into a database (first available as a book) and started writing algorithms to compare proteins across animals and plants, trying to identify their conserved regions and hence their potentially functional domains. Dayhoff was a pioneer in bioinformatics for performing computer-assisted protein sequence alignment. For optimisation reasons, Dayhoff was also responsible for creating the one-letter amino acid code, leading to reduced file sizes.

Years after automatic protein sequencing machines being available based on Edman degradation -- \emph{protein sequenators} as called at the time --, the first-generation DNA sequencing methods were presented: Sanger/dideoxy and Maxam-Gilbert sequencing. % A new method for sequencing DNA, Maxam-Gilbert seq 1977
The first automated DNA sequencing machines by Applied Biosystems (1987) used the Sanger method. Later, the advent of Next-Generation Sequencers (NGS) allowed the massive parallel sequencing of amplified DNA.

% RNA sequencing:
% - Sanger Sequencing of EST
% - SAGE
% - microarray analyses
% - RNA-seq

What is omics? % started from genome and genomics

Transcriptomics is a field that studies the transcriptome -- the set of all RNA transcripts\footnote{Depending on the context, the term \emph{transcriptome} may exclusively refer to the study of mRNA transcripts instead of all RNA transcripts.} ( Charles Auffray (Pietu et al., 1999)) -- using high-throughput technologies that allow to simultaneously analyse the expression of multiple transcripts and employed across a wide array of physiological and disease conditions.

Specifically, the study of AS has been greatly enhanced with the advent of cheaper, high-throughput technologies, since more coverage is required to properly study AS alterations.

% find citations
Transcriptomic studies allow to identify altered phenotype across development stages and pathological subtypes (such as stages of a disease progression), explore the molecular mechanisms underlying a phenotype, pinpoint disease biomarkers, integration with genetic variants and other omics data.

% the importance of process data?
The development and economic feasibility of next-generation sequencing lead multiple consortia to generate a wealth of raw (and processed) sequencing data.

- The Cancer Genome Atlas (TCGA) with molecular and clinical data for more than 30 cancer types, including breast cancer, glioma, 

- The Genotype Tissue Expression (GTEx) project is a database with gene expression data for more than 40 human tissues \cite{lonsdale:2013uo}.

- recount2 has processed RNA-seq data for raw data from Sequencing Read Archive (SRA) \cite{collado-torres:2017uw}.

Even with the increasing economic feasibility of RNA-seq, alternative assays are used when thousands of data are required. Such is the case of L1000, an assay in which the expression of ~12000 genes is estimated based on the values of ~1000 measured genes \cite{subramanian:2017ul}. This particular experiment was the basis for the Connectivity Map (CMap), a database of chemical and genetic perturbations that can be queried with our own up- and down-regulated genes \cite{subramanian:2017ul}. % clue.io?

\section{Bioinformatic apps}

The good, the bad and the ugly.

What is the importance of good user interface/experience?

Reprodutibility

Code optimisation

Benchmarking

Maintained codebase

GitHub

Docker

Interactive, intuitive dashboards
% dashboards for BI

\subsection{Making big data accessible}

What is missing? Make it easier to access big data.

Improve code

Bad UI/UX

Open-source

Automated unit testing

\section{On bioinformatics}

\section{On software engineering}

