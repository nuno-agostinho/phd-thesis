% !TEX root = ../PhD Thesis.tex
\chapter{Introduction}

This is the story of my PhD, a personal journey like no other I have faced before. For you to follow my story, we first have to rewind back to a few years ago. Billions and billions of years ago.

Once upon a time there was a violent, harsh and unwelcoming planet among countless others. Earth was lifeless. But as millions of years went by, it started being home to a complex recipe whose special sauce is still kept secret: the primordial soup. These were the perfect conditions for a young, 500-million-year-old planet to brew life.

And what is life? Although this question is not easy to answer, living organisms as we know them are complex, carbon-based systems composed of nucleic acids, proteins, carbohydrates and lipids. Together with some smaller molecules, these molecules are known as biomolecules (biological molecules) and are crucial for the survival of living organisms.

Amongst biomolecules, two are of particular importance to my tale/story: proteins and nucleic acids. Proteins have many important functions in an organism, including catalysing chemical reactions (enzymes), signalling cellular processes (hormones) and playing a role in the immune system (antigens), among many others. To generate these proteins, the deoxyribonucleic acid (DNA) stores genetic data, a blueprint required to generate proteins.

But life is so complex. How did this all started from the primordial soup? One possible mechanism for the origin of life is based on the idea of an RNA world, where self-replicating RNA proliferated long before DNA and proteins. RNA can both store genetic information (like DNA) and catalyse life-critical chemical reactions (like proteins). DNA and protein may have appeared later as better suited for storing information and catalysing reactions, respectively, leaving RNA in-between.

\section{Alternative splicing}

\subsection{What it is?}

\subsection{In disease context}

\section{Transcriptomics}

\subsection{RNA-seq}

\section{Bioinformatic apps}

The good, the bad and the ugly.

What is the importance of good user interface/experience?

Reprodutibility

Code optimisation

Benchmarking

Maintained codebase

GitHub

Docker

\subsection{Making big data accessible}

What is missing? Make it easier to access big data.