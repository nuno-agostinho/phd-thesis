% !TEX root = ../PhD Thesis.tex
\chapter{Discussion}

Continuing the path I started walking during my MSc, my PhD gave me the opportunity to further develop web apps to be used by the scientific community and to enrich my knowledge about app development and explore multiple new technologies.

\section{psichomics}

Since starting to work in psichomics, I struggled with the lack of guidelines on how to properly design and test interactive bioinformatic web apps. Although there are resources to build generic web apps, the field of bioinformatics could be richer if we better understood how researchers and clinicians explore data to help them find what they are looking for. Also, the lack of a systematic approach to app design is notable in multiple bioinformatics programs in the wild, reflected by popular apps lacking in efficiency and usability, as well as abandoned programs due to completed and/or unrenewed grants.

Ultimately, I think that bioinformaticians that want to create apps to be adopted by the scientific community should be aware of software design to properly write apps following requirement analysis and that are designed for the long-term; data visualisation to design interactive plots that intuitively convey the desired information and allow to conveniently explore the data; and user interface design to improve the user experience and unleash the full potential of the software functionality.

psichomics was the most challenging program I created and the project of my lifetime. It allowed me to develop an app based on multiple topics of my interest (e.g., transcriptomics, data visualisation, web technologies and user interface design), while providing a tool with a graphical interface for alternative splicing and gene expression analysis based on user-provided and public data. % The positive user interaction I received, some of which have led to citations in scientific articles, and the invitation to write the methods book chapter following the publication of psichomics were uplifting and made me feel that psichomics is a good contribution to the field and I can only hope that it stays relevant and useful in the near future.

psichomics is limited in the types of alternative splicing events profiled given that it only supports exon-exon junction read counts when quantifying alternative splicing. Support for other types of reads (exon–intron junction, exon body and intron body counts) would be required to calculate intron retention and alternative 5$'$ and 3$'$ UTR events \cite{braunschweig:2014tr}. This issue is partially mitigated by allowing to import alternative splicing quantifications from user-provided tables and VAST-TOOLS \cite{irimia:2014wt,tapial:2017ui}. For future iterations of psichomics, users have also been requesting support to process the output from other popular alternative splicing quantification tools, including Whippet \cite{sterne-weiler:2018tk} and rMATS \cite{shen:2014tk}.

psichomics was employed in multiple peer-reviewed scientific articles, being used to perform genome-wide identification of differentially spliced events in TCGA data for a review article on alternative splicing in lung cancer \cite{coomer:2019wz}, to assess mutation effects associated with the splicing of specific exons between GTEx brain and skin tissue data alongside VAST-TOOLS and experimental data \cite{baeza-centurion:2019tb}, and to study how \emph{IGF2} over-expression and reduced \emph{SRSF3} splicing activity are negatively associated with the overall survival of liver cancer patients using TCGA data \cite{kumar:2022aa}. The use of psichomics in these publications demonstrates the usefulness of allowing users to access public, pre-processed data to analyse clinical, gene expression and alternative splicing data to complement other bioinformatic and experimental analyses.

\section{cTRAP}

Current \alink{clue.io} tools allow to compare user-provided data with those from CMap, but have some limitations: they only consider a restricted input set of 150 up-regulated and 150 down-regulated genes, are difficult to automate, offer no option to run using local computing resources and have no integration with data from relevant sources that can benefit from similar analyses (e.g., drug sensitivity data from NCI-60, CTRP and GDSC). By overcoming such issues, we expect cTRAP will make it easier for users to identify candidate causal molecular perturbations of phenotypes and compounds, as well as in prioritising targeted therapeutic agents for disease-associated queries.

Thanks to working on psichomics, my knowledge of R was more mature and allowed me to flourish my creativity while developing cTRAP. In the optimisation department, it was challenging to improve cTRAP's runtime, reduce peak memory usage and add multicore support in order to properly perform cTRAP functions as efficiently as possible. This lead to rewriting a lot of the core code in cTRAP 1.4 (\fullref{subsec:ctrap-optim}).

Regarding its user interface, we created graphical interface functions that can be intertwined with R code. Although their practical usefulness may be limited, their modular design made it relatively easy to create the intuitive global interface that culminated into the cTRAP web app. To further test command-line and graphical interfaces of cTRAP, we asked fellow scientists to use them while under cTRAP development and tweaked the interface and internal logic according to their feedback. %Nonetheless, usability tests would help identify bottlenecks and improvements specific to the graphical interface (as previously done for psichomics \cite{saraiva-agostinho:2016vw}), ensuring users can easily perform its advertised functions.

Creating the web app with support for background tasks via Celery and user sessions was also demanding and required a lot of experimentation to design the current implementation, including the creation of the floweRy R package to interact with Flower/Celery, testing the whole integration in the web server and to make all the code and documentation available in cTRAP so users can host their own servers with background task and user session support. Nevertheless, the web app could still benefit from email support to improve the user experience, but it is not a trivial functionality as Celery does not have built-in support for sending emails.

We are currently preparing a manuscript for publication in a peer-review journal. Nonetheless, the early availability of cTRAP in GitHub and Bioconductor led to users asking for the integration of differential gene expression results from single-cell analyses tools, such as Seurat \cite{hao:2021aa}. Such integration with results from popular R packages could promote the use of cTRAP as a downstream analysis tool to further investigate their molecular data.

\section{CompBio}

Creating the app server to host web apps was an enormous challenge, but I am satisfied with the final result. With the exception of R/RStudio and Docker, I had to learn about all of the software stack used (Docker Compose, ShinyProxy, Nginx, Celery\footnote{I started developing the app server at the same time I was researching how to run background tasks for cTRAP. I decided on Celery only after confirming it worked with the app server.}, Plausible, Prometheus, Grafana, etc.) and how these pieces interact amongst each other to properly fine-tune the server to our needs. It was also interesting to configure the server to work in a testing environment so it is easy to test it on any machine before deploying to production.

Our app server currently hosts psichomics and cTRAP, among other web apps from my colleagues. In the future, we intend to host new web apps developed by the lab. In case user demand rises, we may need to increase the available RAM memory and disk space available to the server, which the current iMM infrastructure contemplates.

Although outside of the scope of this work, we decided to make CompBio codebase open-source, allowing other laboratories to repurpose this project, set up their own computing servers and deploy their favourite R and Python web apps. This could make it easier for users to quickly start using the developed web apps without the need to install R packages, Python libraries and their dependencies, as well as avoiding potential installation issues by leveraging Docker images.

\section{PanAShé}

There are many other projects that I have been part of during my PhD but unfortunately did not progress much. However, there is still one that I was a part of and I hope my colleagues will be able to complete: PanAShé.

In a collaborative lab effort, we are developing a Nextflow pipeline to process raw RNA sequencing data from TCGA \cite{chang:2013ww} and GTEx \cite{lonsdale:2013uo} in order to provide processed gene expression and alternative splicing data from samples from multiple normal and diseased tissues. The aims of this project extend those of recount2 \cite{collado-torres:2017uw} and include alternative splicing analysis, as well as a complementary dashboard to help users explore the data in these data sources. We are also considering integrating the data from this project in psichomics in lieu of the limited processed data from the public sources for TCGA and GTEx.

All the software stack in PanAShé is based on Docker images for portability and reproducibility. This means that only Docker and Nextflow are required to run the pipeline. We intend to write a peer-reviewed article regarding this project, as well as share our scripts and processed data with the scientific community as soon as possible.

\section{Conclusion}

With the work I hereby presented, I hope to provide researchers and clinicians with useful tools to analyse gene expression and alternative splicing, predict therapeutic drugs and deploy web apps. Amongst my personal objectives for a PhD, psichomics helped me complete one of them: to create a useful tool to others' research. I can only hope that the rest of my work is as successful as psichomics has been in contributing to science.

As small as all my contributions may have been, these last 4 years were worthy for the prospect of having a (tiny little bit) part in helping unraveling the biological mysteries of this world, along with everything I learned and all the friends I made and danced with along the way.
