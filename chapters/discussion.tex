% !TEX root = ../PhD Thesis.tex
\chapter{Discussion}

\section{PanAShé}

In a collaborative lab effort, we are also developing a Nextflow pipeline to process raw RNA sequencing data from TCGA (Cancer Genome Atlas Research Network et al., 2013) and GTEx (The GTEx Consortium, 2013) in order to provide processed gene expression and alternative splicing data from samples from multiple normal and diseased tissues. The aims of this project extend those of recount2 (Collado-Torres, 2017) and include alternative splicing analysis, as well as a complementary dashboard to help users explore the data in these data sources. We are also considering integrating the data from this project in psichomics in lieu of the limited processed data from the public sources for TCGA and GTEx.

The Nextflow pipeline we are working on is based on Docker images for portability and reproducibility. This means that only Docker and Nextflow are required to be installed in the computer running the pipeline. We intend to write a peer-reviewed article regarding this project, as well as share our scripts and processed data with the scientific community as soon as possible.

\section{Conclusion}

Looking back, I fought my biggest opponents of all time during my PhD. Some days were bright as the sun, others were dark as the night. Some days were spent alongside my friends, others alongside my shadow alone. Some were full of victories, others full of self-doubt.

I fought against and defeated many (software) bugs, yet they keep swarming around me. No matter the quality of the code, no matter the amount of unit testing, no matter the time squatting each pesky bug -- as long as there is code, there will be bugs.

Another boss I struggled with was time: hard to reach as it never stays still. Deadlines are a motivation to strive for and also a cause of distress. It is easy to let oneself be swallowed by each tic, tac, tic, tac. The only way to deal with time is with self-negotiation, a skill otherwise known as time management. I am still learning how to do it efficiently, while doing my best to have moments for everything each day: moments of work, moments of sleep, moments of food and moments of life.

But the biggest enemy of them all was one I knew too well since birth. It was myself: my fears, my anxieties, my insecurities. To this day, I am still learning how to cope with all of them. I believe that, maybe one day, I will be able to convince myself to finally fight alongside me. I can only hope.