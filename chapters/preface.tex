% !TEX root = ../PhD Thesis.tex
\chapter*{Preface}
\addcontentsline{toc}{chapter}{Preface}

There once was a boy who decided to take on a life-changing quest, an adventure where he had to climb the tallest of the mountains and dive into the deepest of the oceans. Although the end goal was not always clear in his mind, his heart was set: he would carry on and stand against everything in his path.

As the boy marched on, he saw a big old pyramid in the far-off distance. Hours and hours went by, yet they seemed to pass in a blink of his eyes. Deep inside the pyramid, within a dark room dimly lit by the weakling flame of a nearby torch, there was a beautiful door featuring a green-jade scarab beetle with its open wings made of a rainbow of precious gems. Above the beetle's raised forelegs laid a gold-carved sun. Although its façade was beautiful, the door was covered in centuries-old dust and its handle has long since forgotten the warm touch of a hand. The boy took a deep breath and opened the door.

Upon entering, he stood inside a big room with a very high ceiling and marvelling hieroglyphs feasting the walls. He was sure it was the finest Egyptian code he has ever seen. As he continued down the room, he started hearing some noises, noises which kept getting closer and closer, like if he was being followed. He quickly turned around. Nothing but darkness. He was alone as far as his eyes could see, yet the noise kept getting closer and closer. Suddenly, the boy looked up and saw countless bugs crawling from cracks above between the hieroglyphs. From one moment to the next, the bugs started swarming him. At first, the boy picked up his sword, swinged it around and tried to deal with all of the bugs, but they started fighting back, eating his patience, his time, his mind. The boy finally decided to simply run away and deal only with the bugs standing between him and the exit. After a tiresome challenge, the boy was able to finally run away from the bug-ridden hell. The boy learned that -- no matter the time squatting each pesky bug -- as long as there is code, there will be bugs.

%In the next room, he started walking and sighing. The more he walked, the more perplexed he was. At each step, the room was feeling smaller and smaller. Was his imagination just playing tricks? As he walked step by step, the walls would get closer and closer. The boy hasted his pace, each step faster than the last one as he was being trapped by the walls until he noticed a trapdoor in the end of the room. When he reached to open the door, he noticed its lock. The walls kept getting closer. Nervous, the boy tried to kick the lock before noticing a key in the floor. He sprinted to the key and back and tried to open the trapdoor. The pressure at the possibility of being squinted like a bug there was too much to bear. Finally, he got it open and slided down the stairs to another room. The boy learned that no matter the pressure, he has to make it in the end.

After many days walking, he entered a big forest where the bushes stood like walls, creating a series of concurrent corridors that lead to different paths. The boy had to continuously decide which corridor to follow, but each decision seemed to him like a bad turn, no matter how right he was. He entered a labyrinth of decisions, where time kept running and running, whether he chose the right paths, the wrong ones or simply stood still wondering which paths to choose. As time went by, he started dashing, and running, and sprinting in the maze, going back and forth through its branches. After a while, he found an exit for that decision-ridden hell. In the end, he learned to deal with the decisions of his past self: to learn from the bad ones and to smile at the good ones. No matter how much he wanted to go back in time, time always carries on and so should he.

As he continued his journey, he saw a small house in the distance. Starved and tired, he entered the house hoping to have some days to pull himself together. Inside, there was a single, almost naked room with mirrors all around the walls. In the middle, a big mirror covered by a linen sheet. The boy got closer to the mirror and removed its cover. A quick glance in the mirror was enough to reveal the boy's reflection and inner thoughts: his fears, his anxieties, his insecurities. The boy gave one step back, afraid of his own image. After all he went through, the boy was fuelled by his negative thoughts. He lowly murmured that there would be no end to his quest, that he would never achieve his goal -- for how does one find something when not even knowing what to look for? Amidst his mournful inner monologue, he heard peaceful voices comforting him. He looked again into the mirror and realised he was not alone, for he knew that many helpful souls that helped him in his path were always there to cheer with him. The burden of this quest was his alone, but that did not mean that he couldn't walk tall aside others. So he ignored his own pessimistic thoughts and continued through his path with the hope of listening to the voices of those he loved once more.

Four years after his first step into this quest, the journey is now coming to an end. His tale ends as many others have: writing his story for others to learn from his past mistakes and glories. And by telling his story, by sharing his experience, by helping other travellers going through his former hurdles, he hopes to contribute to a better world, even if only by a little. The next door he opens will lead to new adventures, but the boy has now learned that no matter the challenges he faces, he will always be welcome in the arms of the ones he loves.
