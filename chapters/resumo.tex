% !TEX root = ../PhD Thesis.tex
\chapter*{Resumo}
\addcontentsline{toc}{chapter}{Resumo}
% 1200 a 1500 palavras

Durante o meu doutoramento no Instituto de Medicina Molecular João Lobo Antunes (iMM), desenvolvi aplicações \emph{web} na área da transcriptómica, estabelecendo-as como recursos gratuitos e de código aberto disponíveis para a comunidade científica. As aplicações foram construídas a partir da linguagem de programação R com elementos gráficos baseados em \emph{Shiny}, um pacote de R para criar aplicações \emph{web}, e podem ser encontradas no Bioconductor, um repositório de pacotes de R.

\section*{psichomics}

A relevância das alterações de \emph{splicing} alternativo em condições fisiológicas e de doença, assim como a viabilidade económica crescente da sequenciação de RNA, têm progressivamente conduzido a um maior interesse no estudo global de \emph{splicing} alternativo \cite{wang:2008wa,tsai:2015ve,danan-gotthold:2015ut,chhibber:2017wm,climente-gonzalez:2017uj} e promovido grandes consórcios a disponibilizar publicamente dados de \emph{splicing} alternativo. Tais consórcios incluem o The Cancer Genome Atlas (TCGA), que cataloga dados clínicos e moleculares de múltiplos tumores humanos \cite{chang:2013ww}; o projecto the Genotype-Tissue Expression (GTEx), que se foca em dados de múltiplos tecidos humanos normais \cite{lonsdale:2013uo}; e o projecto recount2, um recurso com dados pré-processados de mais de 2000 estudos oriundos do Sequence Read Archive (SRA) \cite{collado-torres:2017uw}.

A utilização dos dados pré-processados provenientes destes consórcios garante que os utilizadores não necessitam de armazenar e processar os ficheiros de dados iniciais, requerendo apenas recursos computacionais modestos para analisar expressão génica e \emph{splicing} alternativo. Também é importante notar que os dados humanos não processados podem ser de acesso limitado e requerer um processo formal por razões de privacidade. No entanto, nenhuma ferramenta existente permitia realizar uma análise diferencial de \emph{splicing} alternativo ao nível do transcriptoma com base em dados pré-processados descarregados automaticamente destas fontes públicas e com agrupamento das amostras a partir dos seus metadados.

Assim, desenvolvi o psichomics durante a minha tese de mestrado, uma ferramenta para quantificação, análise e visualização de \emph{splicing} alternativo a partir de dados pré-processados provenientes do TCGA \cite{chang:2013ww}, utilizando uma interface gráfica intuitiva ou a linha de comandos. O psichomics realiza redução de dimensionalidades (por exemplo, através da análise de componentes principais), \emph{splicing} e expressão diferencial e análise de sobrevivência com incorporação de características moleculares e clínicas subjacentes às amostras usadas.

Durante o meu doutoramento, o psichomics foi melhorado de forma a descarregar e processar dados de fontes adicionais -- incluindo dados do GTEx \cite{lonsdale:2013uo}, do recount2 \cite{collado-torres:2017uw} e provenientes do utilizador --, analisar expressão génica, quantificar \emph{splicing} alternativo para 14 espécies diferentes e outras novidades.

% O psichomics está preparado para carregar dados de junções de \emph{splicing} provenientes do utilizador ou de fontes externas, seguido da quantificação de \emph{splicing} alternativo (caso a quantificação em si não seja carregada). Esta quantificação é calculada com base nas \emph{reads} de RNA-seq que alinham contra junções exão-exão e nas coordenadas genómicas (anotação) dos eventos de \emph{splicing} alternativo. A proporção de \emph{reads} alinhadas às junções que suportam a isoforma de inclusão, denominada por \emph{Percent Spliced-In} ou PSI \cite{wang:2008wa}, foi a métrica de quantificação escolhida.

Após a publicação do artigo relativo ao psichomics em 2018 \cite{saraiva-agostinho:2018uq}, fomos convidados a escrever um capítulo metódico publicado no ano de 2020, em que descrevemos como usar o psichomics para analisar diferenças de \emph{splicing} alternativo no contexto de células estaminais humanas \cite{saraiva-agostinho:2020wz}. Desde a sua primeira publicação, o psichomics tem sido usado em vários artigos científicos \cite{coomer:2019wz,baeza-centurion:2019tb,munkley:2019wr,baeza-centurion:2020vb}. Acreditamos que estas citações, juntamente com os comentários positivos que temos vindo a receber dos utilizadores, revelam que vários investigadores e médicos conseguiram usufruir do psichomics para os auxiliar na descoberta de factores de prognóstico e alvos terapêuticos associados a splicing, assim como no avanço do nosso conhecimento sobre como o \emph{splicing} alternativo é regulado em contexto fisiológico e de doença.

\section*{cTRAP}

O Connectivity Map (CMap) é um repositório público de assinaturas transcriptómicas para centenas de perturbações genéticas e farmacológicas em linhas celulares de cancro humanas \cite{subramanian:2017ul}. Ao comparar perfis de expressão génica diferencial com os do CMap, podemos inferir as potenciais causas moleculares das diferenças observadas e compostos que possam promover ou reverter essas alterações.

O CMap and LINCS Unified Environment (\alink{clue.io}) foi desenvolvido como uma colecção de ferramentas intuitivas para explorar os dados do CMap e integrá-los com dados provenientes pelo utilizador \cite{subramanian:2017ul}. No entanto, o \alink{clue.io} limita o máximo número de genes utilizados para comparar com o CMap, expressa os resultados através de um valor de significância fora do padrão, é difícil de automatizar para análises subsequentes e não permite utilizar recursos computacionais locais. Além disso, o \alink{clue.io} não permite actualmente integrar dados de sensibilidade às drogas, que podem auxiliar na identificação de compostos que alvejam células específicas.

Assim, desenvolvemos o cTRAP para identificar potencias perturbações moleculares causais ao comparar resultados completos de expressão génica diferencial providenciados pelo utilizador com aqueles do CMap, assim como prever compostos que promovam ou revertam as diferenças observadas. O cTRAP também permite comparar com associações de expressão génica e sensibilidade às drogas derivadas do NCI-60 \cite{shoemaker:2006wi}, do Cancer Therapeutics Response Portal (CTRP) \cite{seashore-ludlow:2015ws} e do Genomics of Drug Sensitivity in Cancer (GDSC) \cite{yang:2012vk}, para identificar compostos que podem alvejar os fenótipos associados com os perfis de expressão diferencial do utilizador e inclui uma análise similar à do Gene-Set Enrichment Analysis (GSEA) para identificar o enriquecimento de descriptores moleculares para compostos do NCI60 e do CMap. No cTRAP, a similaridade entre resultados de expressão diferencial é medida por \emph{gene set enrichment} \cite{subramanian:2017ul,subramanian:2005wu} e valores de correlação.

O manucripto associado ao cTRAP (do qual eu sou co-primeiro autor e autor co-correspondente) encontra-se em preparação para submissão a uma revista científica \emph{peer-reviewed} internacional. Esperamos que o cTRAP permita aos seus utilizadores identificar potenciais perturbações moleculares causais e compostos para melhor compreender mecanismos biológicos, tal como prever agentes terapêuticos relevantes.

\section*{CompBio: servidor de aplicações \emph{web}}

Ambos o psichomics e o cTRAP apresentam interfaces gráficas para auxiliar os utilizadores a explorar a maioria das funções dos respectivos programas de forma interactiva através de passos simples, devidamente explicados em tutoriais online. Tanto os tutoriais como as próprias ferramentas são constantemente actualizados conforme comentários e pareceres dos próprios utilizadores.

No entanto, apesar das suas interfaces gráficas fáceis de usar, os tradicionais canais de distribuição de pacotes de R, que usamos para distribuir o psichomics e o cTRAP (como o CRAN e o Bioconductor), podem ser dissuasores para quem não se sentir confortável a usar a linha de comandos do R, afastando potenciais utilizadores. Assim, procurei formas alternativas de disponibilizar as nossas aplicações, incluindo conversores de aplicações Shiny em programas normais Electron de computador (infelizmente, as soluções existentes não estão finalizadas), até encontrar o ShinyProxy, um programa de código aberto para hospedar aplicações R/Shiny e Python como aplicações \emph{web} através de instâncias de Docker (\emph{containers}).

Para correr o ShinyProxy, é necessário um servidor para correr o programa constantemente. Dessa forma, para distribuir as ferramentas de forma gratuita e mais acessível requerendo somente a utilização de navegadores de \emph{Internet} modernos, criámos o servidor CompBio para disponibilizar o psichomics, o cTRAP e diversos outros pacotes de R como aplicações \emph{web} -- incluindo programas construídos pelos meus colegas de laboratório (Ageing Atlas, betAS e scStudio). O projecto CompBio está actualmente a correr numa máquina virtual Linux no \emph{cluster} de computação do iMM.

O coração deste projecto é o Docker Compose, um programa que permite gerir múltiplos serviços em simultâneo que correm a partir de imagens Docker e que interagem entre si, incluindo o ShinyProxy, um proxy reverso para as configurações do servidor (Nginx) e programas para correr tarefas em segundo plano (Celery, Redis e Flower), análise de dados do website (Plausible, PostgreSQL e ClickHouse), monitorização de recursos (Prometheus e Grafana) e desenvolvimento (RStudio Web, usado apenas para testar e desenvolver soluções directamente no servidor).

Todo o código deste projecto é de fácil configuração, pode ser facilmente portado para qualquer máquina e está publicamente disponível em \alink{github.com/nuno-agostinho/compbio-app-server}, sendo a plataforma Docker o único requisito de instalação. Para adicionar novas aplicações ao projecto, basta editar as definições do ShinyProxy num ficheiro de texto e disponibilizar a respectiva imagem Docker localmente no servidor ou no repositório Docker Hub. A manutenção do projecto também é fácil, dado que basta actualizar a versão das imagens Docker que se quer usar no projecto e recomeçar quaisquer serviços afectados com o Docker Compose.

Quero aproveitar este momento para pedir uma pausa para sentar e relaxar, para deambular pelo nosso \emph{website} em \alink{compbio.imm.medicina.ulisboa.pt}, para acordar as aplicações \emph{web} que ali pernoitam e para desfrutar da viagem. A página principal é uma galeria de trabalho do nosso laboratório, trabalho o qual tenho um tremendo orgulho em apoiar.\\

\textbf{Palavras-chave:} bioinformática, aplicações \emph{web}, \emph{splicing} alternativo, expressão génica, drogas específicas.

