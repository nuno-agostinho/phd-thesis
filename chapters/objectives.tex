% !TEX root = ../PhD Thesis.tex
\chapter{Objectives}

The aim of the scientific work discussed in this thesis is to develop transcriptomic apps with graphical and command-line interfaces to retrieve and analyse publicly available transcriptomic data from popular repositories, such as TCGA \cite{chang:2013ww}, GTEx \cite{lonsdale:2013uo}, recount2 \cite{collado-torres:2017uw} and CMap \cite{subramanian:2017ul}. Besides, we intend their graphical interfaces to be intuitive, flexible and deployed online as web apps, so that they are used by a wide range of researchers. We think that bridging the gap between big data and researchers with basic computational skills will inspire them to learn more about bioinformatic analyses and thereby potentiate their ability to extract novel biological insights from genome-wide molecular information.

Alternative splicing is involved in multiple cellular processes, with its deregulation being linked to diverse pathologies. The use of pre-processed alternative splicing data from public repositories exempts researchers from storing and processing large raw files that require expensive computational resources. However, until 2018, no tools interactively fetched those public data and allowed their analyses using user-defined sample groups based on associated metadata. Thus, there was a need to develop a program with a graphical interface that allows to quantify, analyse and visualise publicly available alternative splicing data. We used the R programming language to create psichomics \cite{saraiva-agostinho:2018uq,saraiva-agostinho:2020wz}, an open-source R package with a Shiny-based web interface that automatically downloads biological data from TCGA \cite{chang:2013ww}, GTEx \cite{lonsdale:2013uo} and recount2 \cite{collado-torres:2017uw}. By making use of popular R packages like \texttt{edgeR} \cite{robinson:2010wx} and \texttt{limma} \cite{ritchie:2015tm} for gene expression analysis, psichomics interactively performs dimensionality reduction, differential splicing and gene expression and survival analyses with incorporation of molecular and clinical features. % We benchmarked psichomics to check its speed and memory efficiency relative to different data sizes, but also if their quantifications are comparable to the standard alternative splicing quantification tools.

% Based on the work I developed during my MSc thesis \cite{saraiva-agostinho:2016vw}, I continued working on psichomics \cite{saraiva-agostinho:2018uq,saraiva-agostinho:2020wz}, an alternative splicing quantification, analysis and visualisation R package for pre-processed human data from TCGA \cite{chang:2013ww}. During my PhD, psichomics was extended to support more data sources (including GTEx \cite{lonsdale:2013uo}, recount2 \cite{collado-torres:2017uw} and user-provided data), analyse gene expression and support alternative splicing quantification for 14 different species, among other features.

CMap is a public database containing millions of gene expression changes in response to induced molecular and pharmacological perturbations \cite{subramanian:2017ul}. Comparing differential gene expression profiles with those from CMap allows to infer putative molecular causes for the observed differences, as well as compounds that may promote or revert those changes. To facilitate this approach, we strived to build a program to identify potentially causal molecular perturbations by comparing user-provided differential gene expression results with those from CMap, using correlation and gene set enrichment scores. Moreover, the program should also use gene expression/drug sensitivity associations from public databases to identify compounds that may target the phenotypes associated with the user-provided differential expression profiles. We used R/Shiny to create cTRAP, an R package with graphical interface functions to assist users performing the aforementioned features.

% This can be accomplished using user-friendly tools available from the CMap and LINCS Unified Environment (\mbox{\alink{clue.io}}) \cite{subramanian:2017ul}. However, \alink{clue.io} limits the maximum number of input genes for CMap queries, is difficult to automate for downstream analyses and cannot be run using local computing resources. Furthermore, \alink{clue.io} does not currently integrate with drug sensitivity datasets to further assist in pinpointing compounds that selectively target cells \cite{almeida:2019wh}.

%I led the development of cTRAP, an R package to identify candidate causal perturbations from differential gene expression data, as well as predict compounds that may promote or revert them. cTRAP also allows to list putative targeting drugs based on drug sensitivity datasets and includes a GSEA-based analysis of molecular descriptors for compounds from NCI-60 and CMap.

Both psichomics and cTRAP feature web-based graphical interfaces to assist users interactively performing most of their functions following easy steps, properly detailed in online tutorials that are constantly updated according to user feedback. To make Shiny apps accessible via a web browser, we intended to build an open-source and portable codebase to easily deploy web apps using our own custom server. Therefore, I led the development of an app server based on Docker and ShinyProxy to deploy psichomics, cTRAP and multiple R packages as web apps -- including other programs built by my lab colleagues in the lab (voyAGEr, betAS and scStudio). I kindly invite you to pause, sit back and relax, visit our website at \alink{compbio.imm.medicina.ulisboa.pt}, wander through the web apps there and enjoy the journey. The landing page is a gallery of work from our lab that I am deeply proud to support.
